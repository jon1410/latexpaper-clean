%% 	PDF und Links formatieren
\usepackage[pdftex,pagebackref,pdfa]{hyperref}
\pdfcompresslevel=1 %% ZLib Komprimierung: 0==none; 1==fastest; 9==best
\pdfimageresolution=1200 %% Bilder in 600dpi ; Standard ist 300dpi
\pdfminorversion=5

\hypersetup{plainpages=false,
	pdftitle={\mytitle},
	pdfauthor={\myauthor},
	pdfsubject={\mypaper},
	pdfcreator={},
	pdfproducer={},
	pdfkeywords={\mykeywords},
	colorlinks=true,
	linkcolor=black,
	citecolor=black,
	urlcolor=black,
	paper=a4,
	breaklinks=true,
%	linktocpage=true, % verlinkt nur die Seitenzahlen im tableofcontents, da bei Umbruch der Link kaputt gehen kann
	bookmarksopen=true,
	bookmarksnumbered=true,
	bookmarksopenlevel=1,
	pdfmenubar=true,
	pdfwindowui=true,
	pdfview=FitV,
	pdfstartview=Fit,
	pdffitwindow=true
}

% Rückreferenzentext zur Literatur im Quellenverzeichnis (nach hyperref!)
\renewcommand*{\backreftwosep}{ und~}
\renewcommand*{\backreflastsep}{ und~}
\renewcommand*{\backref}[1]{}
\renewcommand*{\backrefalt}[4]{%
\ifcase #1 %
 (nicht zitiert).%
\or
 (zitiert auf Seite~#2).%
\else
 (zitiert auf den Seiten~#2).%
\fi
}

% Glossar / Abkürzungsverzeichnis / Symbolverzeichnis; nach hyperref
\usepackage[
nonumberlist, %keine Seitenzahlen anzeigen
toc,          %Eintrag im Inhaltsverzeichnis
]{glossaries}
